\chapter{Алгоритм работы программы}
\section{Выделение начального базиса}
Для решения ЗЛП исходную матрицу ограничений необходимо представить в каноническом виде и выделить в ней начальный базис. Программа пытается найти базисные вектора в исходной системе уравнений, а при отсутствии таковых, добавляет в систему ограничений балансовые и искусственные переменные (используется метод искусственного базиса).
Данные переменные представляют собой разницу между запасами ресурсов и их потреблением. Коэффициенты при искусственных переменных в исходной функции обозначаются буквой $W$. Подразумевается, что $W$ бесконечно большое число, которое не повлияет на ход нахождения минимума функции.

Алгоритм программы сводит любую задачу к нахождению минимума. Для этого, при максимизации, коэффициенты при целевой функции умножаются на $-1$, что позволяет заменить задачу максимизации минимизацией.

\section{Нахождение оптимального опорного плана}
После выделения начального плана, программа начинает перемещение по вершинам симплекса в поисках оптимального опорного плана. Для этого составляются промежуточные симплекс-таблицы.

В столбцах $P_i$ строки $m+1$ находятся объективно-обусловленные оценки (решение двойственной задачи). Экономическая интерпретация данных оценок может быть различна, но как правило, данные оценки показывают степень важности каждого из ресурсов. В $m+2$ строке находится искусственная часть двойственных оценок, которая при успешном решении задачи обращается в ноль (это может означать, что все искусственные переменные исключены).

Вводимая в базис переменная определяется по максимальному неотрицательному значению в $m+2$ строке. В случае обращения всех значений в не искусственных столбцах $P_i$ строки $m+2$ в ноль, выполняется поиск максимального значения в $m+1$ строке. В случае отсутствия положительных неискусственных элементов в $m+2$ строке и равенству нулю значения в столбце $B$, выполняется поиск максимального элемента в $m+1$ строке над нулевыми элементами $m+2$. Следует учитывать, что искусственная переменная в базис вводится только один раз.

Выводимая из базиса переменная определяется по минимальному положительному элементу в столбце  $\theta$. Отсутствие в столбце положительных значений может говорить о невозможности нахождения оптимального плана. Компонент, находящийся на пересечении выбранного столбца и выбранной строки, называется направляющим.

После нахождения направляющего компонента и смены опорного плана (базиса), поиск решения продолжается вновь, вплоть до нахождения оптимального плана. Критерием остановки может служить отсутствие положительных элементов в $m+1$ строке.

В столбце $B$ $m+1$ строки находится текущее значение целевой функции, которое вычисляется как произведение управляющих переменных в базисе на коэффициенты функции. Сумма произведений искусственных переменных в базисе на $С_j = W$ сохраняется в $m+2$ строке.

\section{Нахождение целочисленного решения}

После нахождения оптимального опорного плана (то есть и решения задачи), программа проверяет условие целочисленности выбранных переменных. В случае, если одна из указанных переменных, не является целочисленной, то программа выполняется отсечение Гомори и смену базиса, после чего поиск оптимального целочисленного решения продолжается вновь.

\section{Вычисление интервалов устойчивости двойственных оценок}
Программа также умеет вычислять интервалы значений $\vec{С}$ и $\vec{B}$, в которых двойственные оценки сохраняют свое значение.

Для нахождения интервалов устойчивости свободных членов $B_m$, из симплекс-таблицы находится обратная матрица (она содержится в столбцах исходного опорного плана канонической задачи) и умножается на приращение вектора $\Delta b_i$. Все элементы решения должны быть неотрицательны, иначе решение будет недопустимым, т.е. базисное решение остаётся оптимальным до тех пор, пока оно допустимое. Для каждого ограничения находится интервал устойчивости, при котором двойственные оценки сохраняют свое значение.

Интервалы устойчивости коэффициентов целевой функции $С_m$ определяются исходя из величины двойственных оценок.