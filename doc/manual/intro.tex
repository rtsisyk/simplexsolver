\chapter{Общая информация}
SimplexSolver решает задачу линейного программирования (ЗЛП).
Задача линейного программирования заключается в нахождении
максимума или минимума линейной целевой функции при заданной системе линейных ограничений.

Основная задача линейного программирования может быть записана в следующем виде:
\begin{equation}
F(\vec{X}) = c_1 x_1 + c_2 x_2 + \hdots + c_n x_n \to max
\end{equation}
\begin{equation}
\label{limits}
\begin{cases}
\begin{matrix}
a_{11} x_1 & + & a_{12} x_2 & + & \hdots &a_{1n} x_n &\le & b_1 \\
a_{21} x_1 & + & a_{22} x_2 & + & \hdots &a_{2n} x_n &\le & b_2 \\
\vdots     & ~ & \vdots     & ~ & \ddots &\vdots     & \vdots & \vdots \\
a_{m1} x_1 & + & a_{m2} x_2 & + & \hdots &a_{mn} x_n &\le & b_m \\
\end{matrix} \\
\end{cases}
\end{equation}

Знаки неравенств \eqref{limits} могут быть как $\le$ и $\ge$, так и $=$.
В последнем случае говорят, что задача представлена в каноническом виде.

Поскольку отрицательные значения $x_{m}$ как правило не имеют экономического смысла,
то к системе ограничений \eqref{limits} часто добавляют условие $x_m \ge 0$. Программа предназначена в первую очередь для учащихся экономических специальностей, поэтому имеет естественное ограничение на не отрицательность искомых переменных.

Нецелые значения $x_{ij}$ иногда также не имеют экономического смысла. Допустим, эти переменные могут показывать количество единиц выпускаемой продукции и дробное значение не уместно в данном случае. Тогда к системе ограничений \eqref{limits} добавляют также условие целочисленности переменных. В этом случае говорят, что это задача целочисленного линейного программирования. Данная возможность предусмотрена в программе.

Обычно условия ЗЛП записывают в виде вектор-строки $\vec{С}$ коэффициентов при целевой функции, матрицы коэффициентов в системе ограничений \eqref{limits} и вектора $\vec{B}$ свободных членов этой системы.