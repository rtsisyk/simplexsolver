\chapter{Установка программы}
Программа поддерживает все основные платформы, в том числе Windows, Unix/X11 и MacOS. Для графического интерфейса используется библиотека Qt 4.5, поэтому данная библиотека должна быть установлена в системе.

\section{Особенности установки под платформу Windows}
Для установки программы следует использовать программу установки SimplexSolver-1.0-setup.exe, находящуюся в дистрибутиве программы и далее следовать инструкция мастера установки.

При установке будет предложено установить также необходимую версию библиотеки Qt, которая включена в дистрибутив программы и данный файл документации.

\section{Сборка из исходных текстов}
Вместе с дистрибутивом программы предоставляются исходные тексты, из которых программа может быть скомпилирована под любую другую платформу, в том числе Linux, Mac OS X, OpenSolaris или FreeBSD.

Для сборки программы необходимо сгенерировать Makefile. Для этого следует использовать команду qmake (входит в состав Qt). После генерации Makefile, программу можно скомпилировать, используя команду gmake (или nmake, зависит от платформы).

Вы можете использовать тексты программы в своих разработках при соблюдении лицензионного соглашения программы.
